%%%%%%%%%%%%%%%%%%%%%%%%%%%%%%%%%%%%%%%%%
% "ModernCV" CV and Cover Letter
% LaTeX Template
% Version 1.1 (9/12/12)
%
% This template has been downloaded from:
% http://www.LaTeXTemplates.com
%
% Original author:
% Xavier Danaux (xdanaux@gmail.com)
%
% License:
% CC BY-NC-SA 3.0 (http://creativecommons.org/licenses/by-nc-sa/3.0/)
%
% Important note:
% This template requires the moderncv.cls and .sty files to be in the same 
% directory as this .tex file. These files provide the resume style and themes 
% used for structuring the document.
%
%%%%%%%%%%%%%%%%%%%%%%%%%%%%%%%%%%%%%%%%%

%----------------------------------------------------------------------------------------
%	PACKAGES AND OTHER DOCUMENT CONFIGURATIONS
%----------------------------------------------------------------------------------------

\documentclass[10pt,a4paper,roman]{moderncv} % Font sizes: 10, 11, or 12; paper sizes: a4paper, letterpaper, a5paper, legalpaper, executivepaper or landscape; font families: sans or roman

\moderncvstyle{banking} % CV theme - options include: 'casual' (default), 'classic', 'oldstyle' and 'banking'
\moderncvcolor{black} % CV color - options include: 'blue' (default), 'orange', 'green', 'red', 'purple', 'grey' and 'black'

\usepackage[adobefonts]{ctex} %中文支持  
\setCJKmainfont{SimSun}  

%\usepackage{lipsum} % Used for inserting dummy 'Lorem ipsum' text into the template

\usepackage[top=1cm,bottom=1cm]{geometry} % Reduce document margins
%\setlength{\hintscolumnwidth}{3cm} % Uncomment to change the width of the dates column
%\setlength{\makecvtitlenamewidth}{10cm} % For the 'classic' style, uncomment to adjust the width of the space allocated to your name

%----------------------------------------------------------------------------------------
%	NAME AND CONTACT INFORMATION SECTION
%----------------------------------------------------------------------------------------

\firstname{万} % Your first name
\familyname{虎} % Your last name

% All information in this block is optional, comment out any lines you don't need
\title{个人简历}
\address{首都师范大学~信息工程学院}{海淀区, 北京市 100048}
\mobile{(+86) 152 10594789}
%\phone{(000) 111 1112}
%\fax{(000) 111 1113}
\email{wanhu@cnu.edu.cn}
\homepage{mengyingchina.github.com}{mengyingchina.github.com} % The first argument is the url for the clickable link, the second argument is the url displayed in the template - this allows special characters to be displayed such as the tilde in this example
%\extrainfo{additional information}
%\photo[70pt][0.4pt]{pictures/picture} % The first bracket is the picture height, the second is the thickness of the frame around the picture (0pt for no frame)
%\quote{"A witty and playful quotation" - John Smith}

%----------------------------------------------------------------------------------------

\begin{document}

\makecvtitle % Print the CV title

%----------------------------------------------------------------------------------------
%	EDUCATION SECTION
%----------------------------------------------------------------------------------------

\section{教育背景}

\cventry{2009--至今}{计算机科学与技术}{首都师范大学}{北京}{\textit{GPA -- 3.93}}{}{访学经历:2012年赴韩国校外访学}  % Arguments not required can be left empty

%----------------------------------------------------------------------------------------
%	WORK EXPERIENCE SECTION
%----------------------------------------------------------------------------------------

\section{项目经历}
%------------------------------------------------
\subsection{``国家大学生创新性实验计划"项目}

\cventry{2010--2011}{项目负责人}{Unix/Linux环境下路由管理转换接口设计与实现}{国家级}{}{
\begin{itemize}
\item 负责在Linux环境下实现软路由功能(基于GNU Zebra),设计实现路由管理转换接口
\item 项目通过学校验收,项目成果在学院部分实验室得到应用
\end{itemize}
}
%------------------------------------------------
%------------------------------------------------
\subsection{``北京市大学生科学研究与创业行动计划"项目}

\cventry{2011--2012}{项目成员}{功耗和温度感知的多核操作系统研究}{市级}{}{
\begin{itemize}
\item 参与设计和实现一种Cache感知的调度算法(CAS),负责算法在Linux环境下的测试和仿真,处理实验数据
\item 项目获得\textbf{``北京市大学生科学研究与创业行动计划"二等奖}
\item 研究论文《面向多核处理器系统的Cache 感知调度算法》发表在中文核心期刊《小型微型计算机系统》
\end{itemize}
}
%------------------------------------------------
%------------------------------------------------
\subsection{首都师范大学``实验室开放基金"项目}

\cventry{2011--2012}{项目负责人}{网络工程创新实验设计——基于Hadoop的海量数据应用研究}{校级}{}{
\begin{itemize}
\item 领导项目小组对Hadoop大规模数据排序算法TeraSort进行分析,探索TeraSort 作为基准测试程序的实际应用
\item 项目被评为首都师范大学``实验室开放基金"优秀项目(学院\textbf{唯一}一个被评为优秀的项目)
\end{itemize}
}
%------------------------------------------------
\subsection{首都师范大学``本科生科学研究与创业行动"项目}

\cventry{2012--至今}{项目负责人}{基于Nios II软核的FIR滤波器的设计}{校级}{}{
\begin{itemize}
	\item 负责解决FIR的IP核的设计,利用Audio ADC/DAC引脚来设计音频输入
	\item 负责解决FIR模块的例化,利用硬件描述语言实现整个音频滤波
\end{itemize}
}
%------------------------------------------------
%------------------------------------------------
\subsection{首都师范大学学位论文\LaTeX 模板开发}
\cventry{2013}{项目负责人}{首都师范大学学位论文(本科生、硕博)\LaTeX 模板}{开源项目}{}{
\begin{itemize}
	\item 开发和维护首都师范大学本科生,硕士生、博士生学位论文\LaTeX 模板
	\item 致力于校内数字排印与阅读的开发和推广
\end{itemize}
}
%----------------------------------------------------------------------------------------
%	AWARDS SECTION
%----------------------------------------------------------------------------------------

\section{荣誉奖励}

\cvitem{2011}{北京市大学生计算机应用大赛移动终端应用创意与程序设计二等奖~团队组长}
\cvitem{2011}{第六届全国信息技术应用水平大赛比赛安卓应用开发团体赛三等奖~团队组长}
\cvitem{2010-2012}{2010、2011、2012 学年度国家励志奖学金}

\end{document}
